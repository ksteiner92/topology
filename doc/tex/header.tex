\usepackage[utf8]{inputenc} % Richtiges anzeigen von Umlauten und quasi allen anderen Schriftzeichen
\usepackage[T1]{fontenc} % Wichtig für alles was mehr als ASCII verwendet
%\usepackage[sc]{mathpazo} % Use the Palatino font
\usepackage{upgreek}
\usepackage{csquotes} % Schöne Anführungsstriche mit \enquote{Text}
\usepackage{amsmath} % Bessere und schönere mathematische Formeln
\usepackage{mathtools} % Noch schönerere mathematische Formeln
\usepackage{amstext} % \text{} Macro in mathematischen Formeln
\usepackage{amsfonts} % Erweiterte Zeichensätze für mathematische Formeln
\usepackage{amssymb} % Spezielle mathematische Symbole.
\usepackage{array} % Matrizen in mathematischen Formeln
\usepackage{textcomp} % Für textmu und textohm etc. um im Fließtext keine Mathematik 
\usepackage{textalpha} % Damit können griechische Zeichen direkt im Text verwendet werden (siehe zeichen.txt)
\usepackage{paralist} % Für compactitem und compactenum
\usepackage{braket} % Für das quantenmechanische Bra-Ket
%\usepackage{geometry} % Seitenränder und Seiteneigenschaften setzen
\usepackage[bottom]{footmisc} % Zwingt Fußnoten an das Ende der Seite
\usepackage[pdftex]{hyperref} % Links richtig anzeigen. Sowohl innerhalb des Dokuments (Fußzeilen, Formeln), als auch ins Internet
\usepackage[ % Biblatex für die Zitate und Referenzen
	backend=bibtex,
	hyperref=true
		]{biblatex}
\addbibresource{ref.bib}
\usepackage{graphicx} % Wichtig für das Einbinden von Grafiken
\usepackage{caption}
\usepackage{subcaption} % Einbinden von mehreren Grafiken in einer figure
\usepackage{float}
\usepackage{cleveref}
\usepackage{graphicx}  
\usepackage{txfonts}  
\usepackage{epstopdf}
\usepackage[miktex]{gnuplottex} % for MiKTeX,`pdflatex -shell-escape` enabled 
\usepackage{fancyhdr}
\usepackage[overload]{empheq}
\usepackage{ulem}
%\usepackage{minted}
\usepackage{listings}
\usepackage{dsfont}
\usepackage{mathtools, cuted}
\usepackage{siunitx} % advanced unit package
\usepackage{titling}
%\usepackage[hmarginratio=1:1,top=10mm,left=12mm, right=12mm, bottom=18mm,columnsep=10pt]{geometry} % Document margins
%\usepackage[hmarginratio=1:1,left=28mm, right=28mm,]{geometry}
%\usepackage[hang, small,labelfont=bf,up,textfont=it,up]{caption} % Custom captions under/above floats in tables or figures
\usepackage{afterpage}
\usepackage{bm}
\usepackage[toc,page]{appendix}
\usepackage{titlesec}


\newcommand{\csubref}[2]{\namecref{#1}~\subref{#1:#2}}
%\renewcommand\thesection{\Roman{section}} % Roman numerals for the sections
%\renewcommand\thesubsection{\roman{subsection}} % roman numerals for subsections
%\titleformat{\section}[block]{\large\scshape\centering}{\thesection.}{1em}{} % Change the look of the section titles
%\titleformat{\subsection}[block]{\large}{\thesubsection.}{1em}{} % Change the look of the section titles
\titleformat{\chapter}{\normalfont\large\bf}{\thechapter.}{18pt}{\huge\bf}


\usepackage{abstract} % Allows abstract customization
\usepackage{blindtext} % Package to generate dummy text throughout this template

\usepackage{fontspec}
\newtheorem{theorem}{Theorem}
%\setmainfont{Verdana}

\renewcommand{\abstractnamefont}{\normalfont\bfseries} % Set the "Abstract" text to bold
\renewcommand{\abstracttextfont}{\normalfont\small\itshape} % Set the abstract itself to small italic

\newcommand*{\widebox}[2][0.5em]{\fbox{\hspace{#1}$\displaystyle #2$\hspace{#1}}}
\newcommand\eqquest{\mathrel{\overset{\makebox[0pt]{\mbox{\normalfont\tiny\sffamily ?}}}{=}}}
\newcommand{\pound}{\operatornamewithlimits{\#}}

\DeclareSIUnit{\calorie}{cal}
\DeclareSIUnit{\Calorie}{\kilo\calorie}
\sisetup{detect-all}

\captionsetup[figure]{font=small,labelfont=small}

\pagestyle{fancy}
\fancyhead{}
\fancyhead[L]{\small \leftmark}
\fancyhead[R]{\small \rightmark}

\hypersetup{ % Setzt einige Werte die in den Eigenschaften des PDF gespeichert sind.
	pdfauthor = {Klaus Steiner},
	pdftitle = {Topology},
	pdfdisplaydoctitle = true,
	colorlinks = false, % Für Druck auf "false" setzen!
}
